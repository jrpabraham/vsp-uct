{
\def\sym#1{\ifmmode^{#1}\else\(^{#1}\)\fi}
\begin{tabular}{l*{10}{SSSSS}}
\toprule
          &\multicolumn{6}{c}{Spillover Effects}                      &\multicolumn{2}{c}{Lee Bounds}&\multicolumn{2}{c}{Horowitz-Manski Bounds}\\\cmidrule(lr){2-7}\cmidrule(lr){8-9}\cmidrule(lr){10-11}
          &\multicolumn{1}{c}{(1)}&\multicolumn{1}{c}{(2)}&\multicolumn{1}{c}{(3)}&\multicolumn{1}{c}{(4)}&\multicolumn{1}{c}{(5)}&\multicolumn{1}{c}{(6)}&\multicolumn{1}{c}{(7)}&\multicolumn{1}{c}{(8)}&\multicolumn{1}{c}{(9)}&\multicolumn{1}{c}{(10)}\\
          &\multicolumn{1}{c}{\specialcell{All HH\\Estimate}}&\multicolumn{1}{c}{\specialcell{All HH\\estimate}}&\multicolumn{1}{c}{\specialcell{Thatched\\estimate}}&\multicolumn{1}{c}{\specialcell{Thatched\\estimate}}&\multicolumn{1}{c}{\specialcell{Test (1)=(3)\\\emph{p}-value}}&\multicolumn{1}{c}{\specialcell{Test (2)=(4)\\\emph{p}-value}}&\multicolumn{1}{c}{\specialcell{Lower}}&\multicolumn{1}{c}{\specialcell{Upper}}&\multicolumn{1}{c}{\specialcell{Lower}}&\multicolumn{1}{c}{\specialcell{Upper}}\\
\midrule
Includes controls&     {No}&    {Yes}&     {No}&    {Yes}&     {No}&    {Yes}&     {No}&     {No}&     {No}&     {No}\\
\midrule Log cortisol (no controls)&    -0.06&    -0.06&    -0.06&    -0.06&     0.98&     0.88&    -0.10&    -0.05&-0.08$^{*}$&    -0.04\\
          &   (0.05)&   (0.05)&   (0.05)&   (0.05)&         &         &   (0.08)&   (0.05)&   (0.05)&   (0.05)\\
Log cortisol (with controls)&    -0.05&    -0.05&    -0.05&    -0.05&     0.92&     0.99&    -0.08&    -0.03&    -0.07&    -0.03\\
          &   (0.05)&   (0.05)&   (0.05)&   (0.05)&         &         &   (0.07)&   (0.05)&   (0.05)&   (0.05)\\
Depression (CESD)&    -0.23&    -0.25&    -0.20&    -0.20&     0.89&     0.79&    -0.26&    -0.20&    -0.43&    -0.01\\
          &   (0.79)&   (0.81)&   (0.82)&   (0.85)&         &         &   (0.64)&   (0.54)&   (0.51)&   (0.51)\\
Worries   &     0.08&     0.07&     0.06&     0.05&     0.34&     0.26&     0.08&     0.08&     0.06&0.10$^{*}$\\
          &   (0.08)&   (0.08)&   (0.08)&   (0.08)&         &         &   (0.07)&   (0.06)&   (0.05)&   (0.05)\\
Stress (Cohen)&     0.07&     0.08&     0.08&     0.09&     0.71&     0.64&     0.07&0.11$^{**}$&     0.05&0.09$^{*}$\\
          &   (0.07)&   (0.07)&   (0.07)&   (0.07)&         &         &   (0.07)&   (0.05)&   (0.05)&   (0.05)\\
Happiness (WVS)&0.11$^{*}$&     0.10&     0.10&     0.09&     0.72&     0.69&     0.11&0.21$^{***}$&     0.08&0.13$^{**}$\\
          &   (0.06)&   (0.06)&   (0.07)&   (0.07)&         &         &   (0.10)&   (0.04)&   (0.05)&   (0.05)\\
Life satisfaction (WVS)&     0.02&     0.02&    -0.00&    -0.00&     0.17&     0.20&     0.02&     0.03&     0.01&     0.05\\
          &   (0.08)&   (0.08)&   (0.08)&   (0.08)&         &         &   (0.06)&   (0.07)&   (0.05)&   (0.05)\\
Trust (WVS)&    -0.08&    -0.09&    -0.09&    -0.10&     0.62&     0.66&    -0.08&    -0.08&-0.09$^{*}$&    -0.06\\
          &   (0.07)&   (0.07)&   (0.07)&   (0.07)&         &         &   (0.06)&   (0.06)&   (0.05)&   (0.05)\\
Locus of control&    -0.09&    -0.09&-0.11$^{*}$&-0.12$^{*}$&     0.21&     0.17&    -0.09&    -0.09&-0.11$^{**}$&    -0.06\\
          &   (0.06)&   (0.06)&   (0.06)&   (0.06)&         &         &   (0.06)&   (0.07)&   (0.05)&   (0.05)\\
Optimism (Scheier)&     0.09&     0.10&0.11$^{*}$&0.12$^{*}$&     0.22&     0.22&     0.08&     0.09&     0.06&0.11$^{**}$\\
          &   (0.07)&   (0.07)&   (0.07)&   (0.07)&         &         &   (0.08)&   (0.06)&   (0.05)&   (0.05)\\
Self-esteem (Rosenberg)&    -0.03&    -0.02&    -0.03&    -0.02&     0.92&     0.78&    -0.03&    -0.02&    -0.05&    -0.00\\
          &   (0.07)&   (0.07)&   (0.07)&   (0.07)&         &         &   (0.08)&   (0.07)&   (0.05)&   (0.05)\\
Psychological well-being index&     0.03&     0.03&     0.03&     0.02&     0.77&     0.71&     0.03&     0.04&     0.01&     0.05\\
          &   (0.07)&   (0.07)&   (0.07)&   (0.07)&         &         &   (0.07)&   (0.06)&   (0.05)&   (0.05)\\
\midrule Joint test (\emph{p}-value)&     0.35&     0.26&     0.18&     0.11&         &         &         &         &         &         \\
\bottomrule
\end{tabular}
}
