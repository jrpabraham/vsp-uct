\documentclass[aspectratio=169]{beamer}

\usetheme{boxes}
\useinnertheme{circles}

\definecolor{UCSD}{RGB}{24, 43, 73}
\definecolor{Chicago}{RGB}{128, 0, 0}
\definecolor{Forest}{RGB}{101, 121, 48}
\definecolor{Hooker}{RGB}{73, 121, 107}
\definecolor{Gamboge}{RGB}{228, 155, 15}
\definecolor{Prussian}{RGB}{0, 49, 83}
\definecolor{Metro}{RGB}{35, 54, 58}
\definecolor{Devil}{RGB}{115, 39, 43}
\definecolor{Shade}{RGB}{250, 250, 250}

\usecolortheme[named=UCSD]{structure}
\setbeamercolor{background canvas}{bg=Shade}

% \setbeamertemplate{footline} % To remove the footer line in all slides uncomment this line
\setbeamertemplate{footline}[page number] % To replace the footer line in all slides with a simple slide count uncomment this line
\setbeamertemplate{navigation symbols}{} % To remove the navigation symbols from the bottom of all slides uncomment this line
\setbeamertemplate{bibliography item}{}

\setbeamertemplate{frametitle}{%
    \usebeamerfont{frametitle} \vspace{0.5em} \insertframetitle%
    \vphantom{g}% To avoid fluctuations per frame
    \vspace{0.5em} \hrule% Uncomment to see desired effect, without a full-width hrule
    \par\hspace*{-\dimexpr0.5\textwidth-0.5\textwidth}\rule[0.5\baselineskip]{\textwidth}{0.4pt}
    \par\vspace*{-\baselineskip}% <-- reduce vertical space after rule
}

\setbeamersize{text margin left=0.2in,text margin right=0.2in}

\usepackage[T1]{fontenc}
\usepackage{inconsolata}
\usepackage[sfdefault,light]{roboto}
% \usepackage[default]{lato}
\usepackage{amsmath, amssymb, amsthm, mathrsfs, bbm}
\usepackage{threeparttable, adjustbox, tabularx, longtable, booktabs}
\usepackage{pdflscape}
\usepackage{placeins}
\usepackage{caption}
\usepackage{hyperref}
\usepackage{graphicx}
\usepackage{listings}
\usepackage{color}

\newcommand{\inv}[1]{#1^{-1}}
\newcommand{\iid}{\text{i.i.d.}}
\newcommand{\bmat}[1]{\begin{bmatrix} #1 \end{bmatrix}}
\newcommand{\asconv}{\xrightarrow{a.s.}}
\newcommand{\pconv}{\xrightarrow{p}}
\newcommand{\dconv}{\xrightarrow{d}}
\newcommand{\msconv}{\xrightarrow{m.s.}}
\newcommand{\liminfty}{\lim_{n \to \infty}}
\newcommand*\diff{\mathop{}\!\mathrm{d}}
\newcommand{\lhood}{\mathcal{L}}
\newcommand{\var}{\mathrm{Var}}
\newcommand{\ev}{\mathrm{E}}
\renewcommand{\vec}[1]{\mathbf{#1}}
\newcommand{\specialcell}[2][c]{\begin{tabular}[#1]{@{}c@{}}#2\end{tabular}}

\newtheorem*{proposition}{Proposition}
\newtheorem*{claim}{Claim}

\newenvironment{wideitemize}{\itemize\addtolength{\itemsep}{10pt}}{\enditemize}
\newenvironment{wideenumerate}{\enumerate\addtolength{\itemsep}{10pt}}{\endenumerate}

\let\oldforall\forall
\let\forall\undefined
\DeclareMathOperator{\forall}{\oldforall}
\DeclareMathOperator*{\argmax}{arg\,max}
\DeclareMathOperator*{\argmin}{arg\,min}

\lstset{
  basicstyle=\footnotesize\ttfamily,
  columns=fixed,
  fontadjust=true,
  basewidth=0.5em
}

\usepackage[backend=bibtex, style=authortitle, citestyle=authoryear-icomp, url=false]{biblatex}
\addbibresource{Giving.bib}

\title{Heterogeneous Spillovers in Unconditional Cash Transfers}
\author[Abraham, Bechhofer, Kim]{Justin Abraham \and Nathaniel Bechhofer \and Minki Kim}
\institute{University of California, San Diego}
\date{\today}

\begin{document}

\begin{frame}
	\titlepage
\end{frame}

\begin{frame}[fragile] \frametitle{Introduction}

    \begin{wideenumerate}
        \item Stuff
        \item More stuff
        \item Yet more
    \end{wideenumerate}

\end{frame}

\begin{frame}[fragile] \frametitle{Overview}

    \begin{wideitemize}
        \item Blah
        \item Blah
        \item Blah
    \end{wideitemize}

\end{frame}

\begin{frame}[fragile] \frametitle{Overview}

    Testing alternative theories of philantropic choice on one dimension: an information shock

    \begin{wideitemize}
        \item Something
    \end{wideitemize}

\end{frame}

\begin{frame}[fragile] \frametitle{Literature}

    \begin{wideitemize}
        \item Much of the literature has focused on allocation between self and others and motivations thereof
        \item \textcite{null_warm_2011} looks at allocation across various matching rates (prices) and identifies between warm glow and risk aversion
        \item \textcite{caviola_evaluability_2014} examines the ``evaluability bias'' with joint evaluation of charities
        \item Some work on this (Exley, Small, Brock, Krawczyk, Karlan, Null)
        \item Lots of advocacy (Singer, McAskill, Baron, etc.)
        \item GiveWell's own evaluations of ``money moved''
        \item Goal: synthesize a model to be able to rationalize these obsevations
    \end{wideitemize}

\end{frame}

\begin{frame}[fragile] \frametitle{Some buzz on the subject}

		\begin{wideitemize}
			\item Landsburg: https://web.archive.org/web/20110201203530/http://www.slate.com/id/2034/
			\item Hoskin: https://www.givingwhatwecan.org/post/2013/11/should-you-only-donate-to-one-charity/
            \item Harford: https://slate.com/culture/2006/10/the-economic-case-against-charity.html
			\item EA: https://concepts.effectivealtruism.org/concepts/philanthropic-diversification/
            \item Donor coordination: https://claviger.net/concept-for-donor-coordination.html, https://blog.givewell.org/2014/12/02/donor-coordination-and-the-givers-dilemma/
            \item 80K: https://80000hours.org/2016/02/the-value-of-coordination/, https://80000hours.org/articles/coordination/
            \item Ben Kuhn: https://www.benkuhn.net/how-many-causes
            \item Yudkowsky: https://www.lesswrong.com/posts/3p3CYauiX8oLjmwRF/purchase-fuzzies-and-utilons-separately
            \item Forbes: https://www.forbes.com/sites/vanguardcharitable/2017/11/01/many-donors-diversify-their-charitable-portfolio-among-several-giving-tools/
		\end{wideitemize}

\end{frame}

\begin{frame}[fragile] \frametitle{Competing explanations}

    Broad themes: Consequentialist vs. procedural, preferences vs. environment, strategic vs. unilateral

	\begin{enumerate}
        \item Pure altruists hedging uncertainty over distributional outcomes
		\item Impure altruists obtain utility from additional charities
        \item Equality concerns for charities
        \item Overestimate curvature of the social welfare function or impact of donation
        \item Rule utilitarianism/donor coordination: extrapolate best donation to all and infer diminishing marginal returns
        \item Rule utilitarianism/donor coordination: free-rider problem reintroduced with kinks due to funding targets
        \item Donor coordination: want to be donor of last resort/non-fungibilities
        \item Staggered entry of causes/NGOs and the presence of transition costs -> differences between long run and short run philantropy
        \item New information and the presence of transaction costs
        \item Errors, salience, heuristics, narrow bracketing etc.
        \item Incommensurability
        \item Independence violations and joint evaluation of alternatives
	\end{enumerate}

    But how can we distinguish between these? What other phenomena do they have to rationalize?

\end{frame}

\begin{frame}[fragile] \frametitle{Public goods game}

    \begin{wideitemize}
        \item Is this even the right exercise to model things like redistribution?
        \item Maybe it's good enough--redistribution is a consequence of public goods
        \item Is it strategic or can we imagine a continuum of donors? Clearly no
    \end{wideitemize}

\end{frame}

\begin{frame}[fragile] \frametitle{Implications for behavior and welfare}

    What behaviors can we predict under each model? Welfare implications?

    \begin{wideitemize}
        \item Information update: no effect on impure altruists, concentrating effect on EU maximizers, diversification if donor coordination strategic game! Are we back to free riding? No just diversification
        \item Telling people how pivotal they are can result in lower donations (this is not the case under pure altruism)
        \item Neutrality:
        \item Optimal information policy of NGOs
        \item NGO incentives to increase/decrease funding gaps
        \item Should charities give not just snapshots of marginal benefit but the global marginal benefit?
        \item Do the effects go away when you can coordinate? -> donor coordination problem
        \item Welfare comparisons of redistribution schemes
        \item Does not address the extensive margin
        \item What is the role of meta-charities/donor advised funds?
        \item Salience heuristic: people will become more purely altruistic
    \end{wideitemize}

\end{frame}

\begin{frame}[fragile] \frametitle{Empirical strategy}

    \begin{wideitemize}
        \item Expand VCM games to include multiple public goods with risky, non-linear technologies
        \item Concave technologies vs. funding caps and targets
        \item Lab contributions using actual charities
        \item Exploit Good Ventures policy changes to test https://blog.givewell.org/2014/12/02/donor-coordination-and-the-givers-dilemma/
        \item PSID has amount donated to different causes
        \item Lab can control for motives and targets and just focus on uncertainty
        \item IRS itemizations: https://nccs.urban.org/, https://www.urban.org/research/publication/profiles-individual-charitable-contributions-state-2013
        \item Is there data for volunteer work and time use?
    \end{wideitemize}

\end{frame}

\begin{frame}[fragile] \frametitle{Notes}

	\begin{wideitemize}
		\item Check handbook chapter on underdiversification
	\end{wideitemize}

\end{frame}

\begin{frame}[fragile] \frametitle{Duplicate Observations}

	\tiny { \begin{verbatim}
		. count
		  7,845

		. duplicates report survey_id

		Duplicates in terms of survey_id

		--------------------------------------
		   copies | observations       surplus
		----------+---------------------------
		        1 |         7811             0
		        2 |           34            17
		--------------------------------------

		. duplicates list survey_id

		Duplicates in terms of survey_id

		  +---------------------------------+
		  | group:   obs:         survey_id |
		  |---------------------------------|
		  |      1    534   601010202004027 |
		  |      1    535   601010202004027 |
		  |      2    697   601010204009011 |
		  |      2    698   601010204009011 |
		  |      3   1888   601030203004030 |
		  |      3   1894   601030203004030 |
	\end{verbatim} }

\end{frame}

\begin{frame}[fragile] \frametitle{Math}

    \begin{equation*}
    y_i = \beta_0 + \beta_1 S_v + \beta_2 D^\text{Abs.}_i + \beta_3 S_v \times D^\text{Abs.}_i + \varepsilon_i
    \end{equation*}

\end{frame}

\begin{frame}[fragile] \frametitle{References}

    \printbibliography

\end{frame}

\begin{frame}[fragile] \frametitle{Scratch}

    Diversification
        Why diversify? Risk aversion and warm glow giving
        Extending Null/Karlan with information intervention and multiple choices
        Price of giving might not be the same as effectiveness
        Null, Exley, Niehaus: no demand for information

    Joint vs separate evaluation of charities
        Motivation: People think charity evaluation should just impact distribution but can also affect net donations (except for knife edge case)
        Outcomes: gross amount donated, spread, convergence
        Treatments: between vs. within subject charities, risk and certainty, information prepost, private vs. public
        Theory: Effect of salience (more distributional behavior in joint groups), BT preferences
        Fisman, Null do this but doesn't document the degree of bias (assuming independence)
        Over which attribute? Price/quality, effectiveness/visibility
        But what if there are alternative mechanisms?
        Add arm where dominated charity not framed by effectiveness
        Add arms where deviation from reference increases
        Add arms where number of dominated charities increase
        Need to control for confusion?
        Does this mean no GARP? ie done already? It means non-separability
        Provide per unit cost from GiveWell data

    X Do effectiveness elasticities mirror price elasticities? (utility over money received vs. final outcome)
        Are they the same thing?
        In the lab effectively, but not in the field where there is uncertainty
        EG show subsidies/rebates are different so no reason to think so
        Must control for uncertainty in the former
        Present different costs per unit according to GiveWell
        Extend/mirror EG experiment
        Possible difference: with BT preferences and visibility, effectiveness exhibits no overjustification effect
        Need to calibrate which amounts where this phenomenon occurs
        Instead of manipulating visibility, manipulate the inputs over which others' inference is made about donors
        What is effectivness to people? Could just be definition
        The key thing is whether incentives accrue to donors directly or not

    Basic documentation of warm glow: over total amount, donate at all, number of charities, charity features, visibility

    Choi-style consistency experiments for others

    Aversion to low-rated charities may be because of excuses or because it reveals type

    Use an online representative sample and not WEIRD sample

    Simple tests on the price effects of charities: what if we examine allocation across charities? the matching might be a signal of quality or could be excuse driven (Exley)

        Andreoni-Miller, Fisman already kind of do this fitting utility
        Fisman distinguishes altruism and social preferences and fits utility
            Why don't they talk about warm-glow? Subsume into self-other preferences? BT preferences are beyond this study

        Karlan-List find elasticity in the field and kink at 0
        Eckel-Grossman find elasticity with matches not subsidies in the lab/field
        Rondeau-List finds elasticity: challenge vs. matching doesnt matter, lab vs. field matters: it's a signal about quality
        Huck-Rasul distinguish between received/donated price elasticities, identify importance of lead donors, optimal fundraising
        Null estimates elasticities using a menu of charities

        **excuse driven responses (Exley, Dana) or signalling (BT) could deflate previous elasticities because its between self and other
        **some things we need to acount for: excuses, warm glow -> BT reputation, effectiveness
        **Can be cleaner budgets than the Null study
        **Can try to distinguish warm glow and risk aversion
        **positive vs. negative domain
        **what does this say about preferences? what can this predict?

        impure altruism should attenuate both types of cross elasticities? Exley: excuses could inflate them actually (in the negative domain)

        Weirdness with matches and rebates (Eckel Grossman): could be signal of the leader about the quality, warm glow of giving via BT prefs, confusion
            Davis 2005 takes care of confusion
            Add a disincentive arm because may not be symmetric (Andreoni cold prickle is about taking money from public good)
            Vary public/private indep of rebate/match to get at BT prefs
        Meer everything cool when prices are related to actual costs not matches etc.
        If true elasticities are low: what can we attribute to this? Null paper
        Implies uncertainty regarding impact is demanded
        Exley: vary the level of inference can be made about donor type and check donations
        **impure altruism not commensurate with leader as signal

\end{frame}

\end{document}
