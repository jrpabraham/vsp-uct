 
\documentclass[14pt, aspectratio=169]{beamer}
\setbeamertemplate{navigation symbols}{} 
\mode<presentation>
{
  \usetheme{Boadilla}
  \usecolortheme{dolphin}
  % or ...

  \setbeamercovered{transparent}
  % or whatever (possibly just delete it)
}

\usepackage[english]{babel}
% or whatever
\usepackage[utf8]{inputenc}
% or whatever
\usepackage{times}
\usepackage[T1]{fontenc}
\usepackage{verbatim}



% Or whatever. Note that the encoding and the font should match. If T1
% does not look nice, try deleting the line with the fontenc.

\title[Heteregenous Spillovers in Unconditional Cash Transfer] % (optional, use only with long paper titles)
{Heterogenous Spillovers in Unconditional Cash Transfer}

\author[Abraham, Bechhofer, and Kim] % (optional, use only with lots of authors)
{Abraham, Bechhofer, and Kim }
% - Use the \inst{?} command only if the authors have different
%   affiliation.


%\date[2018.10.16] % (optional)
%{2018.10.16}

\subject{Talks}
% This is only inserted into the PDF information catalog. Can be left
% out.


% If you have a file called "university-logo-filename.xxx", where xxx
% is a graphic format that can be processed by latex or pdflatex,
% resp., then you can add a logo as follows:

% \pgfdeclareimage[height=0.5cm]{university-logo}{university-logo-filename}
% \logo{\pgfuseimage{university-logo}}

% Delete this, if you do not want the table of contents to pop up at
% the beginning of each subsection:
%\AtBeginSubsection[]
%{
%  \begin{frame}<beamer>{Outline}
%    \tableofcontents[currentsection,hideallsubsections]
%  \end{frame}
%}

%\AtBeginSection[]{
%	\begin{frame}
%		\vfill
%		\centering
%		\begin{beamercolorbox}[sep=8pt,center,shadow=true,rounded=true]{title}
%			\usebeamerfont{title}\insertsectionhead\par%
%		\end{beamercolorbox}
%		\vfill
%	\end{frame}
% }

% If you wish to uncover everything in a step-wise fashion, uncomment
% the following command:

%\beamerdefaultoverlayspecification{<+->}
\usepackage{amsmath}
\usepackage{bm}
\usepackage{accents}
\usepackage{float}
\usepackage{stackengine}
\usepackage{rotating}
\usepackage{adjustbox}
\usepackage{graphicx}
\usepackage{kotex}

\setbeamertemplate{itemize items}[default]
\setbeamertemplate{enumerate items}[default]

\let\OLDitemize\itemize
\renewcommand\itemize{\OLDitemize\addtolength{\itemsep}{10pt}}

\let\OLDenumerate\enumerate
\renewcommand\enumerate{\OLDenumerate\addtolength{\itemsep}{10pt}}



\begin{document}

\begin{frame}
  \titlepage
\end{frame}

\begin{frame}{Motivation}
\begin{itemize}
	\item Househofer and Sharpiro (2016) RCT: villages where villagers were given transfers saw even those who did not receive transfers obtain spillover benefits.
	\item Our question: \\ Does everyone experiences the same amount of spillover? 
\end{itemize}
\end{frame}

\begin{frame}{Intervention}
content...
\end{frame}

\begin{frame}{Identifying heterogeneity}
\begin{itemize}
	\item Heterogeneity in linear spillover effects:
	\begin{equation*} \label{eq:interaction}
	Y_{i,v} = \beta_0 + \beta_1 S_v + \beta_2 D_{i,v} + \beta_3 S_v \times  D_{i,v} + \varepsilon_{i,v}
	\end{equation*}
	
	\item $Y_{i,v}$: Outcome variable of interest 
    \item $S_v$: Indicator for living in a treatment village 
    \item $D_{i,v}$: Measure of demographic distance of individual $i$
 
\end{itemize}
\end{frame}

\begin{frame}{Measuring Demographic Distance}
\begin{itemize}
	\item Absolute distance
	$$D_{i,v}  = \frac{|Y_{i,v,t=0} - \bar Y_{v,t=0}|}{\text{SD}_v} $$
	\item Squared deviations from village averages
	$$ D^2_{i,v}= \frac{(Y_{i,v,t=0} - \bar Y_{v,t=0})^2}{\text{SD}_v} $$
	\item Mahalanobis measure
	$$D^\text{M.}_{i,v} = \sqrt{( X_i - {\bar X})' {\hat S}^{-1}_v ( X_i - {\bar X})}  $$ 
\end{itemize}
\end{frame}

\begin{frame}{Results}
content...
\end{frame}

\begin{frame}{Conclusion}
content...
\end{frame}



\end{document}


