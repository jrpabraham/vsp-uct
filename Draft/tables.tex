\documentclass[11pt]{article}

\usepackage[margin=1.5in]{geometry}
\usepackage{amsmath, amssymb, amsthm, mathrsfs}
\usepackage{lmodern, inconsolata}
\usepackage{bbm}
\usepackage{fancyhdr}
\usepackage{caption, graphicx}
\usepackage{listings}
\usepackage{color}
\usepackage{enumitem}
\usepackage{booktabs, threeparttable, adjustbox, tabularx, longtable}
\usepackage{hyperref}
\usepackage{soul}

\newcommand{\inv}[1]{#1^{-1}}
\newcommand{\iid}{\text{i.i.d.}}
\newcommand{\bmat}[1]{\begin{bmatrix} #1 \end{bmatrix}}
\newcommand{\asconv}{\xrightarrow{a.s.}}
\newcommand{\pconv}{\xrightarrow{p}}
\newcommand{\dconv}{\xrightarrow{d}}
\newcommand{\msconv}{\xrightarrow{m.s.}}
\newcommand{\liminfty}{\lim_{n \to \infty}}
\newcommand*\diff{\mathop{}\!\mathrm{d}}
\newcommand{\lhood}{\mathcal{L}}
\renewcommand{\vec}[1]{\mathbf{#1}}

\newtheorem*{proposition}{Proposition}
\newtheorem*{claim}{Claim}

\let\oldforall\forall
\let\forall\undefined
\DeclareMathOperator{\forall}{\oldforall}
\DeclareMathOperator{\ev}{E}
\DeclareMathOperator{\var}{Var}
\DeclareMathOperator*{\argmax}{arg\,max}
\DeclareMathOperator*{\argmin}{arg\,min}

\lstset{
  basicstyle=\footnotesize\ttfamily,
  columns=fixed,
  fontadjust=true,
  basewidth=0.5em
}

\newcommand{\specialcell}[2][c]{\begin{tabular}[#1]{@{}c@{}}#2\end{tabular}}

\usepackage[backend=bibtex, style=authortitle, citestyle=authoryear-icomp, url=false]{biblatex}
\addbibresource{citations.bib}

\linespread{1}
\pagestyle{fancy}
\lhead{Very Short Paper Proposal}
\rhead{Abraham, Bechhofer, Kim}

\begin{document}

\begin{table}
\centering
\caption{Spillover effects by squared distance from village means}
\maxsizebox*{\textwidth}{\textheight}{
\begin{threeparttable}
{
\def\sym#1{\ifmmode^{#1}\else\(^{#1}\)\fi}
\begin{tabular}{l*{5}{c}}
\toprule
          &\multicolumn{1}{c}{Interaction}&\multicolumn{1}{c}{Sq. distance}&\multicolumn{1}{c}{\specialcell{Treated village}}&\multicolumn{1}{c}{\specialcell{Control mean\\(Std. dev.)}}&\multicolumn{1}{c}{Obs.}\\
\midrule
\midrule Value of non-land assets (USD)&-0.436$^{***}$&0.550$^{***}$&94.834$^{***}$&   384.05&      901\\
          &  (0.052)&  (0.043)& (21.403)& (298.69)&         \\
Non-durable expenditure (USD)&-0.434$^{***}$&0.472$^{***}$&21.569$^{***}$&   165.38&      901\\
          &  (0.036)&  (0.025)&  (7.409)&  (90.90)&         \\
Total revenue, monthly (USD)&-0.470$^{***}$&0.499$^{***}$&24.317$^{***}$&    52.66&      901\\
          &  (0.018)&  (0.017)&  (4.650)&  (95.22)&         \\
Food security index&0.310$^{***}$&-0.419$^{***}$&-0.279$^{***}$&    -0.06&      901\\
          &  (0.061)&  (0.031)&  (0.085)&   (1.26)&         \\
Health index&    0.023&   -0.052&   -0.068&     0.06&      901\\
          &  (0.099)&  (0.077)&  (0.092)&   (1.06)&         \\
Education index&   -0.034&0.215$^{***}$&    0.086&    -0.01&      750\\
          &  (0.096)&  (0.080)&  (0.091)&   (1.03)&         \\
Psychological well-being index&    0.052&    0.003&   -0.017&    -0.03&     1396\\
          &  (0.068)&  (0.060)&  (0.086)&   (0.98)&         \\
Female empowerment index&0.406$^{***}$&-0.424$^{***}$&-0.217$^{**}$&    -0.21&      661\\
          &  (0.055)&  (0.019)&  (0.091)&   (1.15)&         \\
\bottomrule
\end{tabular}
}

\begin{tablenotes}[flushleft] \footnotesize \item \emph{Notes}:
The unit of observation is the household for all outcome variables except for the psychological variables index, where it is the individual. The sample is restricted to co-habitating couples for the female empowerment index, and households with school-age children for the education index. All columns include village-level fixed effects, control for baseline outcomes, and cluster standard errors at the household level. * denotes significance at 10 pct., ** at 5 pct., and *** at 1 pct. level.
\end{tablenotes}
\end{threeparttable}
}
\end{table}

\begin{table}
\centering
\caption{Spillover effects by absolute distance from village means}
\maxsizebox*{\textwidth}{\textheight}{
\begin{threeparttable}
{
\def\sym#1{\ifmmode^{#1}\else\(^{#1}\)\fi}
\begin{tabular}{l*{5}{c}}
\toprule
          &\multicolumn{1}{c}{Interaction}&\multicolumn{1}{c}{Abs. distance}&\multicolumn{1}{c}{\specialcell{Treated village}}&\multicolumn{1}{c}{\specialcell{Control mean\\(Std. dev.)}}&\multicolumn{1}{c}{Obs.}\\
\midrule
\midrule Value of non-land assets (USD)&-111.103$^{***}$&203.299$^{***}$&92.200$^{**}$&   384.05&      901\\
          & (41.440)& (31.743)& (37.393)& (298.69)&         \\
Non-durable expenditure (USD)&-49.941$^{***}$&52.684$^{***}$&32.360$^{***}$&   165.38&      901\\
          & (10.184)&  (6.917)&  (9.504)&  (90.90)&         \\
Total revenue, monthly (USD)&-68.286$^{***}$&98.174$^{***}$&48.735$^{***}$&    52.66&      901\\
          & (17.755)& (11.947)& (11.143)&  (95.22)&         \\
Food security index&0.473$^{***}$&-0.628$^{***}$&-0.307$^{**}$&    -0.06&      901\\
          &  (0.166)&  (0.140)&  (0.129)&   (1.26)&         \\
Health index&   -0.059&   -0.015&    0.000&     0.06&      901\\
          &  (0.153)&  (0.117)&  (0.121)&   (1.06)&         \\
Education index&    0.158&    0.035&   -0.055&    -0.01&      750\\
          &  (0.167)&  (0.122)&  (0.122)&   (1.03)&         \\
Psychological well-being index&    0.066&    0.014&   -0.022&    -0.03&     1396\\
          &  (0.103)&  (0.078)&  (0.102)&   (0.98)&         \\
Female empowerment index&0.913$^{***}$&-0.917$^{***}$&-0.513$^{***}$&    -0.21&      661\\
          &  (0.155)&  (0.109)&  (0.129)&   (1.15)&         \\
\bottomrule
\end{tabular}
}

\begin{tablenotes}[flushleft] \footnotesize \item \emph{Notes}:
The unit of observation is the household for all outcome variables except for the psychological variables index, where it is the individual. The sample is restricted to co-habitating couples for the female empowerment index, and households with school-age children for the education index. All columns include village-level fixed effects, control for baseline outcomes, and cluster standard errors at the household level. * denotes significance at 10 pct., ** at 5 pct., and *** at 1 pct. level.
\end{tablenotes}
\end{threeparttable}
}
\end{table}

%
% The main specification replicates Haushofer and Shapiro (2016) and is estimated between treatment households and households in treatment villages who did not receive the cash transfer. $T_{h,v}$ is an indicator for assignment to receive cash. $W_{h,v}$ is the pre-treatment level of the outcome variable $Y_{h,v}$. $\alpha$ are village fixed effects. $\varepsilon_{h,v}$ is the idiosyncratic error term. Standard errors are clustered at the household level.
%
%     \[ Y_{h,v} = \alpha + \beta_0 T_{h,v} + \beta_1 W_{h,v} + \varepsilon_{h,v} \]
%
% The current study examines heterogeneous treatment effects by interacting $T_{h,v}$ with a pre-treatment characteristic $X_{h,v}$. $\alpha$ are village fixed effects. $\eta_{h,v}$ is the idiosyncratic error term. Standard errors are clustered at the household level.
%
%     \[ Y_{h,v} = \alpha + \gamma_0 T_{h,v} + \gamma_1 X_{h,v} + \gamma_2 (T_{h,v} \times X_{h,v}) + \eta_{h,v} \]
%
% \begin{table}
% \centering
% \caption{Treatment effects}
% \maxsizebox*{\textwidth}{\textheight}{
% \begin{threeparttable}
% {
\def\sym#1{\ifmmode^{#1}\else\(^{#1}\)\fi}
\begin{tabular}{l*{6}{c}}
\toprule
          &\multicolumn{1}{c}{\specialcell{Control\\mean (SD)}}&\multicolumn{1}{c}{\specialcell{Treatment\\effect}}&\multicolumn{1}{c}{\specialcell{Female\\recipient}}&\multicolumn{1}{c}{\specialcell{Monthly\\transfer}}&\multicolumn{1}{c}{\specialcell{Large\\transfer}}&\multicolumn{1}{c}{Obs.}\\
\midrule
Value of non-land assets (USD)&   494.80&301.51$^{***}$&   -79.46&-91.85$^{**}$&279.18$^{***}$&      940\\
          & (415.32)&  (27.25)&  (50.38)&  (45.92)&  (49.09)&         \\
Non-durable expenditure (USD)&   157.61&35.66$^{***}$&    -2.00&    -4.20&21.25$^{**}$&      940\\
          &  (82.18)&   (5.85)&  (10.28)&  (10.71)&  (10.49)&         \\
Total revenue, monthly (USD)&    48.98&16.15$^{***}$&     5.41&    16.33&    -2.44&      940\\
          &  (90.52)&   (5.88)&  (10.61)&  (11.07)&   (8.87)&         \\
Food security index&     0.00&0.26$^{***}$&     0.06&0.26$^{**}$&0.18$^{*}$&      940\\
          &   (1.00)&   (0.06)&   (0.09)&   (0.11)&   (0.10)&         \\
Health index&    -0.00&    -0.03&     0.10&     0.01&    -0.09&      940\\
          &   (1.00)&   (0.06)&   (0.09)&   (0.10)&   (0.09)&         \\
Education index&     0.00&     0.08&     0.06&    -0.05&     0.05&      823\\
          &   (1.00)&   (0.06)&   (0.09)&   (0.10)&   (0.09)&         \\
Psychological well-being index&    -0.00&0.26$^{***}$&0.14$^{*}$&     0.01&0.26$^{***}$&     1474\\
          &   (1.00)&   (0.05)&   (0.08)&   (0.08)&   (0.08)&         \\
Female empowerment index&    -0.00&    -0.01&0.17$^{*}$&     0.05&0.22$^{**}$&      698\\
          &   (1.00)&   (0.07)&   (0.10)&   (0.12)&   (0.11)&         \\
\bottomrule
\end{tabular}
}

% \begin{tablenotes}[flushleft] \footnotesize \item \emph{Notes}:
% OLS estimates of treatment effects. Outcome variables are listed on the left. Higher values correspond to ``positive'' outcomes. For each outcome variable, we report the coefficients of interest and their standard errors in parentheses. FWER-corrected $p$-values are shown in brackets. Column (1) reports the mean and standard deviation of the spillover group, column (2) the basic treatment effect, i.e. comparing treatment households to control households within villages. Column (3) reports the relative treatment effect of transferring to the female compared to the male; column (4) the relative effect of monthly compared to lump-sum transfers; and column (5) that of large compared to small transfers. The unit of observation is the household for all outcome variables except for the psychological variables index, where it is the individual. The sample is restricted to co-habitating couples for the female empowerment index, and households with school-age children for the education index. The comparison of monthly to lump-sum transfers excludes large transfer recipient households, and that for male vs. female recipients excludes single-headed households. All columns include village-level fixed effects, control for baseline outcomes, and cluster standard errors at the household level. * denotes significance at 10 pct., ** at 5 pct., and *** at 1 pct. level.
% \end{tablenotes}
% \end{threeparttable}
% }
% \end{table}
%
% \begin{table}
% \centering
% \caption{Heterogeneous treatment effects -- Secondary education}
% \maxsizebox*{\textwidth}{\textheight}{
% \begin{threeparttable}
% {
\def\sym#1{\ifmmode^{#1}\else\(^{#1}\)\fi}
\begin{tabular}{l*{5}{c}}
\toprule
          &\multicolumn{1}{c}{\specialcell{Control\\mean (SD)}}&\multicolumn{1}{c}{Interaction}&\multicolumn{1}{c}{Treatment}&\multicolumn{1}{c}{\specialcell{Received\\secondary edu.}}&\multicolumn{1}{c}{Obs.}\\
\midrule
Value of non-land assets (USD)&   494.80&   -28.64&319.74$^{***}$&    28.57&      940\\
          & (415.32)&  (58.12)&  (44.59)&  (37.24)&         \\
Non-durable expenditure (USD)&   157.61&     4.15&32.46$^{***}$&14.36$^{*}$&      940\\
          &  (82.18)&  (12.60)&   (9.93)&   (7.47)&         \\
Total revenue, monthly (USD)&    48.98&    -6.47&19.90$^{**}$&17.32$^{**}$&      940\\
          &  (90.52)&  (11.82)&   (8.30)&   (7.03)&         \\
Food security index&     0.00&    -0.01&0.26$^{**}$&     0.11&      940\\
          &   (1.00)&   (0.15)&   (0.12)&   (0.10)&         \\
Health index&    -0.00&0.32$^{**}$&-0.25$^{**}$&    -0.00&      940\\
          &   (1.00)&   (0.14)&   (0.12)&   (0.11)&         \\
Education index&     0.00&    -0.03&     0.10&     0.07&      823\\
          &   (1.00)&   (0.13)&   (0.10)&   (0.09)&         \\
Psychological well-being index&    -0.00&     0.04&0.23$^{**}$&     0.13&     1474\\
          &   (1.00)&   (0.12)&   (0.10)&   (0.08)&         \\
Female empowerment index&    -0.00&0.42$^{**}$&-0.31$^{**}$&    -0.04&      698\\
          &   (1.00)&   (0.17)&   (0.15)&   (0.12)&         \\
\bottomrule
\end{tabular}
}

% \begin{tablenotes}[flushleft] \footnotesize \item \emph{Notes}:
% he unit of observation is the household for all outcome variables except for the psychological variables index, where it is the individual. The sample is restricted to co-habitating couples for the female empowerment index, and households with school-age children for the education index. All columns include village-level fixed effects, control for baseline outcomes, and cluster standard errors at the household level. * denotes significance at 10 pct., ** at 5 pct., and *** at 1 pct. level.
% \end{tablenotes}
% \end{threeparttable}
% }
% \end{table}
%
% \begin{table}
% \centering
% \caption{Heterogeneous treatment effects -- Marital status}
% \maxsizebox*{\textwidth}{\textheight}{
% \begin{threeparttable}
% {
\def\sym#1{\ifmmode^{#1}\else\(^{#1}\)\fi}
\begin{tabular}{l*{5}{c}}
\toprule
          &\multicolumn{1}{c}{\specialcell{Control\\mean (SD)}}&\multicolumn{1}{c}{Interaction}&\multicolumn{1}{c}{Treatment}&\multicolumn{1}{c}{Married}&\multicolumn{1}{c}{Obs.}\\
\midrule
Value of non-land assets (USD)&   494.80&   -45.47&337.87$^{***}$&    68.23&      940\\
          & (415.32)&  (63.51)&  (54.92)&  (44.67)&         \\
Non-durable expenditure (USD)&   157.61&    12.53&25.89$^{**}$&22.87$^{***}$&      940\\
          &  (82.18)&  (13.58)&  (11.90)&   (8.24)&         \\
Total revenue, monthly (USD)&    48.98&    22.10&    -1.19&     0.66&      940\\
          &  (90.52)&  (14.04)&  (12.11)&  (10.41)&         \\
Food security index&     0.00&0.50$^{***}$&    -0.14&     0.04&      940\\
          &   (1.00)&   (0.17)&   (0.15)&   (0.12)&         \\
Health index&    -0.00&     0.24&    -0.22&    -0.03&      940\\
          &   (1.00)&   (0.19)&   (0.18)&   (0.14)&         \\
Education index&     0.00&     0.11&    -0.01&    -0.07&      823\\
          &   (1.00)&   (0.19)&   (0.17)&   (0.14)&         \\
Psychological well-being index&    -0.00&0.48$^{***}$&    -0.15&-0.22$^{**}$&     1474\\
          &   (1.00)&   (0.15)&   (0.14)&   (0.11)&         \\
Female empowerment index&    -0.00&     0.03&    -0.03&     0.15&      698\\
          &   (1.00)&   (0.33)&   (0.32)&   (0.24)&         \\
\bottomrule
\end{tabular}
}

% \begin{tablenotes}[flushleft] \footnotesize \item \emph{Notes}:
% he unit of observation is the household for all outcome variables except for the psychological variables index, where it is the individual. The sample is restricted to co-habitating couples for the female empowerment index, and households with school-age children for the education index. All columns include village-level fixed effects, control for baseline outcomes, and cluster standard errors at the household level. * denotes significance at 10 pct., ** at 5 pct., and *** at 1 pct. level.
% \end{tablenotes}
% \end{threeparttable}
% }
% \end{table}

\printbibliography

\end{document}
